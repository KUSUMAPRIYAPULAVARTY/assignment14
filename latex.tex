\documentclass[journal,12pt,twocolumn]{IEEEtran}

\usepackage{setspace}
\usepackage{gensymb}

\singlespacing


\usepackage[cmex10]{amsmath}

\usepackage{amsthm}

\usepackage{mathrsfs}
\usepackage{txfonts}
\usepackage{stfloats}
\usepackage{bm}
\usepackage{cite}
\usepackage{cases}
\usepackage{subfig}

\usepackage{longtable}
\usepackage{multirow}

\usepackage{enumitem}
\usepackage{mathtools}
\usepackage{steinmetz}
\usepackage{tikz}
\usepackage{circuitikz}
\usepackage{verbatim}
\usepackage{tfrupee}
\usepackage[breaklinks=true]{hyperref}
\usepackage{graphicx}
\usepackage{tkz-euclide}

\usetikzlibrary{calc,math}
\usepackage{listings}
    \usepackage{color}                                            %%
    \usepackage{array}                                            %%
    \usepackage{longtable}                                        %%
    \usepackage{calc}                                             %%
    \usepackage{multirow}                                         %%
    \usepackage{hhline}                                           %%
    \usepackage{ifthen}                                           %%
    \usepackage{lscape}     
\usepackage{multicol}
\usepackage{chngcntr}

\DeclareMathOperator*{\Res}{Res}

\renewcommand\thesection{\arabic{section}}
\renewcommand\thesubsection{\thesection.\arabic{subsection}}
\renewcommand\thesubsubsection{\thesubsection.\arabic{subsubsection}}

\renewcommand\thesectiondis{\arabic{section}}
\renewcommand\thesubsectiondis{\thesectiondis.\arabic{subsection}}
\renewcommand\thesubsubsectiondis{\thesubsectiondis.\arabic{subsubsection}}


\hyphenation{op-tical net-works semi-conduc-tor}
\def\inputGnumericTable{}                                 %%

\lstset{
%language=C,
frame=single, 
breaklines=true,
columns=fullflexible
}
\begin{document}


\newtheorem{theorem}{Theorem}[section]
\newtheorem{problem}{Problem}
\newtheorem{proposition}{Proposition}[section]
\newtheorem{lemma}{Lemma}[section]
\newtheorem{corollary}[theorem]{Corollary}
\newtheorem{example}{Example}[section]
\newtheorem{definition}[problem]{Definition}

\newcommand{\BEQA}{\begin{eqnarray}}
\newcommand{\EEQA}{\end{eqnarray}}
\newcommand{\define}{\stackrel{\triangle}{=}}
\bibliographystyle{IEEEtran}

\providecommand{\mbf}{\mathbf}
\providecommand{\pr}[1]{\ensuremath{\Pr\left(#1\right)}}
\providecommand{\qfunc}[1]{\ensuremath{Q\left(#1\right)}}
\providecommand{\sbrak}[1]{\ensuremath{{}\left[#1\right]}}
\providecommand{\lsbrak}[1]{\ensuremath{{}\left[#1\right.}}
\providecommand{\rsbrak}[1]{\ensuremath{{}\left.#1\right]}}
\providecommand{\brak}[1]{\ensuremath{\left(#1\right)}}
\providecommand{\lbrak}[1]{\ensuremath{\left(#1\right.}}
\providecommand{\rbrak}[1]{\ensuremath{\left.#1\right)}}
\providecommand{\cbrak}[1]{\ensuremath{\left\{#1\right\}}}
\providecommand{\lcbrak}[1]{\ensuremath{\left\{#1\right.}}
\providecommand{\rcbrak}[1]{\ensuremath{\left.#1\right\}}}
\theoremstyle{remark}
\newtheorem{rem}{Remark}
\newcommand{\sgn}{\mathop{\mathrm{sgn}}}
\providecommand{\abs}[1]{\left\vert#1\right\vert}
\providecommand{\res}[1]{\Res\displaylimits_{#1}} 
\providecommand{\norm}[1]{\left\lVert#1\right\rVert}
%\providecommand{\norm}[1]{\lVert#1\rVert}
\providecommand{\mtx}[1]{\mathbf{#1}}
\providecommand{\mean}[1]{E\left[ #1 \right]}
\providecommand{\fourier}{\overset{\mathcal{F}}{ \rightleftharpoons}}
%\providecommand{\hilbert}{\overset{\mathcal{H}}{ \rightleftharpoons}}
\providecommand{\system}{\overset{\mathcal{H}}{ \longleftrightarrow}}
	%\newcommand{\solution}[2]{\textbf{Solution:}{#1}}
\newcommand{\solution}{\noindent \textbf{Solution: }}
\newcommand{\cosec}{\,\text{cosec}\,}
\providecommand{\dec}[2]{\ensuremath{\overset{#1}{\underset{#2}{\gtrless}}}}
\newcommand{\myvec}[1]{\ensuremath{\begin{pmatrix}#1\end{pmatrix}}}
\newcommand{\mydet}[1]{\ensuremath{\begin{vmatrix}#1\end{vmatrix}}}

\numberwithin{equation}{subsection}

\makeatletter
\@addtoreset{figure}{problem}
\makeatother
\let\StandardTheFigure\thefigure
\let\vec\mathbf

\renewcommand{\thefigure}{\theproblem}

\def\putbox#1#2#3{\makebox[0in][l]{\makebox[#1][l]{}\raisebox{\baselineskip}[0in][0in]{\raisebox{#2}[0in][0in]{#3}}}}
     \def\rightbox#1{\makebox[0in][r]{#1}}
     \def\centbox#1{\makebox[0in]{#1}}
     \def\topbox#1{\raisebox{-\baselineskip}[0in][0in]{#1}}
     \def\midbox#1{\raisebox{-0.5\baselineskip}[0in][0in]{#1}}
\vspace{3cm}
\title{Assignment 14}
\author{KUSUMA PRIYA\\EE20MTECH11007}

\maketitle
\newpage

\bigskip
\renewcommand{\thefigure}{\theenumi}
\renewcommand{\thetable}{\theenumi}
Download codes from 
%
\begin{lstlisting}
https://github.com/KUSUMAPRIYAPULAVARTY/assignment14
\end{lstlisting}
%
 
 \section{QUESTION}
Let $p,m,n$ be positive integers and $\mathbbf{F}$ a field.Let $\vec{V}$ be the space of $m \times n$ matrices over $\mathbbf{F}$ and $\vec{W}$ the space of $p \times n$ matrices over $\mathbbf{F}$.Let $\vec{B}$ be a fixed $p \times m$ matrix and let $\mathbbf{T}$ be the linear transformation from $\vec{V}$ into $\vec{W}$ defined by $\mathbbf{T}(\vec{A})=\vec{B}\vec{A}$.Prove that $\mathbbf{T}$ is invertible if and only if $p=m$ and $\vec{B}$ is an invertible $m \times m$ matrix. 
\end{align}

%

\section{Solution}
\begin{align}
    \mathbbf{T}(\vec{A})=\vec{B}\vec{A}
\end{align}
So, $\vec{B}$ is the transformation matrix.\\
Transformation matrix is invertible if
\begin{enumerate}
    \item $\mathbbf{T}$ is one to one mapping,that is for a matrix $\vec{A} \in \vec{V}$ there exists a unique $\vec{C} \in \vec{W}$ such that
    \begin{align}
        \mathbbf{T}(\vec{A})=\vec{B}\vec{A}=\vec{C} \label{1}
    \end{align}
    \item $\mathbbf{T}$ must be onto, that is range($\vec{B}$)=$\vec{W}$ that is,
    \begin{align}
      \text{rank}(\vec{B}) =p \label{3}
    \end{align}
\end{enumerate}
Let us assume that $\mathbbf{T}$ is invertible. Therefore,there exists an operator $\mathbbf{U}$ from $\vec{W}$ to $\vec{V}$ such that
\begin{align}
    \mathbbf{U}(\vec{C})=\vec{U}\vec{C}=\vec{A} \label{2}
\end{align}
where $\vec{U}_{m \times p}$ is the inverse transformation matrix.\\
If $\vec{B}$ is one to one, onto then from \eqref{1} and \eqref{2} $\vec{U}$ is also one to one which implies range($\vec{U}$) is $\vec{V}$
\begin{align}
    \text{rank}(\vec{U})=m \label{4}
\end{align}
But the relation between transformation matrix and its inverse is
\begin{align}
   \vec{U}=\vec{B}^{-1} 
\end{align}
which means that the rank of $\vec{B},\vec{U}$ must be same.\\
 So,from \eqref{3},\eqref{4} we can write 
\begin{align}
    p=m
\end{align}
So, if $\mathbbf{T}$ is invertible linear transformation, then matrix $\vec{B}$ is invertible and $p=m$
\\
\\
Now,consider it the other way.We consider $p=m$ and $\vec{B}$ is an invertible $m \times m$ matrix and wish to prove that $\mathbbf{T}$ is an invertible transformation.
Since $\vec{B}$ is invertible, rank of $\vec{B}$ is $m$ so $\mathbbf{T}$ is onto.\\
To prove $\mathbbf{T}$ is one to one, consider the matrix $\vec{A} \in \vec{V}$.Since $\vec{B}^{-1}$ exists, we can write
\begin{align}
    \vec{B}^{-1}(\mathbbf{T}(\vec{A}))= \vec{B}^{-1}(\vec{B}\vec{A})\\=(\vec{B}^{-1}\vec{B})\vec{A}\\=\vec{A}
\end{align}
So, $\vec{B}^{-1}$ is the transformation matrix that maps every $\mathbbf{T}(\vec{A}) \in \vec{W}$ to $\vec{A} \in \vec{V}$\\
So, $\mathbbf{T}$ is invertible.\\
Hence,from both the cases of consideration, $\mathbbf{T}$ is invertible if and only if $p=m$ and $\vec{B}$ is an invertible $m \times m$ matrix.
\end{document}

