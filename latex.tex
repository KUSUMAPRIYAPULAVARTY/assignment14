\documentclass[journal,12pt,twocolumn]{IEEEtran}

\usepackage{setspace}
\usepackage{gensymb}

\singlespacing


\usepackage[cmex10]{amsmath}

\usepackage{amsthm}

\usepackage{mathrsfs}
\usepackage{txfonts}
\usepackage{stfloats}
\usepackage{bm}
\usepackage{cite}
\usepackage{cases}
\usepackage{subfig}

\usepackage{longtable}
\usepackage{multirow}

\usepackage{enumitem}
\usepackage{mathtools}
\usepackage{steinmetz}
\usepackage{tikz}
\usepackage{circuitikz}
\usepackage{verbatim}
\usepackage{tfrupee}
\usepackage[breaklinks=true]{hyperref}
\usepackage{graphicx}
\usepackage{tkz-euclide}

\usetikzlibrary{calc,math}
\usepackage{listings}
    \usepackage{color}                                            %%
    \usepackage{array}                                            %%
    \usepackage{longtable}                                        %%
    \usepackage{calc}                                             %%
    \usepackage{multirow}                                         %%
    \usepackage{hhline}                                           %%
    \usepackage{ifthen}                                           %%
    \usepackage{lscape}     
\usepackage{multicol}
\usepackage{chngcntr}

\DeclareMathOperator*{\Res}{Res}

\renewcommand\thesection{\arabic{section}}
\renewcommand\thesubsection{\thesection.\arabic{subsection}}
\renewcommand\thesubsubsection{\thesubsection.\arabic{subsubsection}}

\renewcommand\thesectiondis{\arabic{section}}
\renewcommand\thesubsectiondis{\thesectiondis.\arabic{subsection}}
\renewcommand\thesubsubsectiondis{\thesubsectiondis.\arabic{subsubsection}}


\hyphenation{op-tical net-works semi-conduc-tor}
\def\inputGnumericTable{}                                 %%

\lstset{
%language=C,
frame=single, 
breaklines=true,
columns=fullflexible
}
\begin{document}


\newtheorem{theorem}{Theorem}[section]
\newtheorem{problem}{Problem}
\newtheorem{proposition}{Proposition}[section]
\newtheorem{lemma}{Lemma}[section]
\newtheorem{corollary}[theorem]{Corollary}
\newtheorem{example}{Example}[section]
\newtheorem{definition}[problem]{Definition}

\newcommand{\BEQA}{\begin{eqnarray}}
\newcommand{\EEQA}{\end{eqnarray}}
\newcommand{\define}{\stackrel{\triangle}{=}}
\bibliographystyle{IEEEtran}

\providecommand{\mbf}{\mathbf}
\providecommand{\pr}[1]{\ensuremath{\Pr\left(#1\right)}}
\providecommand{\qfunc}[1]{\ensuremath{Q\left(#1\right)}}
\providecommand{\sbrak}[1]{\ensuremath{{}\left[#1\right]}}
\providecommand{\lsbrak}[1]{\ensuremath{{}\left[#1\right.}}
\providecommand{\rsbrak}[1]{\ensuremath{{}\left.#1\right]}}
\providecommand{\brak}[1]{\ensuremath{\left(#1\right)}}
\providecommand{\lbrak}[1]{\ensuremath{\left(#1\right.}}
\providecommand{\rbrak}[1]{\ensuremath{\left.#1\right)}}
\providecommand{\cbrak}[1]{\ensuremath{\left\{#1\right\}}}
\providecommand{\lcbrak}[1]{\ensuremath{\left\{#1\right.}}
\providecommand{\rcbrak}[1]{\ensuremath{\left.#1\right\}}}
\theoremstyle{remark}
\newtheorem{rem}{Remark}
\newcommand{\sgn}{\mathop{\mathrm{sgn}}}
\providecommand{\abs}[1]{\left\vert#1\right\vert}
\providecommand{\res}[1]{\Res\displaylimits_{#1}} 
\providecommand{\norm}[1]{\left\lVert#1\right\rVert}
%\providecommand{\norm}[1]{\lVert#1\rVert}
\providecommand{\mtx}[1]{\mathbf{#1}}
\providecommand{\mean}[1]{E\left[ #1 \right]}
\providecommand{\fourier}{\overset{\mathcal{F}}{ \rightleftharpoons}}
%\providecommand{\hilbert}{\overset{\mathcal{H}}{ \rightleftharpoons}}
\providecommand{\system}{\overset{\mathcal{H}}{ \longleftrightarrow}}
	%\newcommand{\solution}[2]{\textbf{Solution:}{#1}}
\newcommand{\solution}{\noindent \textbf{Solution: }}
\newcommand{\cosec}{\,\text{cosec}\,}
\providecommand{\dec}[2]{\ensuremath{\overset{#1}{\underset{#2}{\gtrless}}}}
\newcommand{\myvec}[1]{\ensuremath{\begin{pmatrix}#1\end{pmatrix}}}
\newcommand{\mydet}[1]{\ensuremath{\begin{vmatrix}#1\end{vmatrix}}}

\numberwithin{equation}{subsection}

\makeatletter
\@addtoreset{figure}{problem}
\makeatother
\let\StandardTheFigure\thefigure
\let\vec\mathbf

\renewcommand{\thefigure}{\theproblem}

\def\putbox#1#2#3{\makebox[0in][l]{\makebox[#1][l]{}\raisebox{\baselineskip}[0in][0in]{\raisebox{#2}[0in][0in]{#3}}}}
     \def\rightbox#1{\makebox[0in][r]{#1}}
     \def\centbox#1{\makebox[0in]{#1}}
     \def\topbox#1{\raisebox{-\baselineskip}[0in][0in]{#1}}
     \def\midbox#1{\raisebox{-0.5\baselineskip}[0in][0in]{#1}}
\vspace{3cm}
\title{Assignment 14}
\author{KUSUMA PRIYA\\EE20MTECH11007}

\maketitle
\newpage

\bigskip
\renewcommand{\thefigure}{\theenumi}
\renewcommand{\thetable}{\theenumi}
Download codes from 
%
\begin{lstlisting}
https://github.com/KUSUMAPRIYAPULAVARTY/assignment14
\end{lstlisting}
%
 
 \section{QUESTION}
Let $p,m,n$ be positive integers and $\mathbbf{F}$ a field.Let $\vec{V}$ be the space of $m \times n$ matrices over $\mathbbf{F}$ and $\vec{W}$ the space of $p \times n$ matrices over $\mathbbf{F}$.Let $\vec{B}$ be a fixed $p \times m$ matrix and let $\mathbbf{T}$ be the linear transformation from $\vec{V}$ into $\vec{W}$ defined by $\mathbbf{T}(\vec{A})=\vec{B}\vec{A}$.Prove that $\mathbbf{T}$ is invertible if and only if $p=m$ and $\vec{B}$ is an invertible $m \times m$ matrix. 
\end{align}

%

\section{Solution}
\begin{table}[!ht]
\centering
\begin{tabular}{|p{3.7cm}|p{4cm}|}
\hline
\textbf{Parameter}&\textbf{Description}\\
\hline
$p,m,n$&Positive integers\\
\hline
$\mathbbf{F}$&Field\\
\hline
$\vec{V}$&Space of $m\times n$ matrices over $\mathbbf{F}$\\
\hline
$\vec{W}$&Space of $p\times n$ matrices over $\mathbbf{F}$\\
\hline
$\vec{B}$&Fixed $p\times m$ matrix\\
\hline
Linear transformation  $\mathbbf{T}:\vec{V} \rightarrow \vec{W}$&$ \mathbbf{T}(\vec{A})=\vec{B}\vec{A}$\\
\hline
\end{tabular}
\caption{Input Parameters}
\end{table}
\begin{align}
    \mathbbf{T}(\vec{A})=\vec{B}\vec{A}
\end{align}
So, $\vec{B}$ is the transformation matrix.\\
$\vec{B}$ is invertible if
\begin{enumerate}
    \item $\mathbbf{T}$ is one to one mapping,that is
    \begin{align}
       \vec{B}\vec{A}=\vec{B}\vec{A'}\\
       \implies \vec{A}=\vec{A'}
    \end{align}
    \item $\mathbbf{T}$ must be onto, that is range($\vec{B}$)=$\vec{W}$ 
\end{enumerate}
\subsection{Case 1}
Let us assume that $\mathbbf{T}$ is invertible with inverse transformation $\mathbbf{T}_1$ from $\vec{W}$ to $\vec{V}$\\
Therefore,for $\vec{C} \in \vec{W}$ 
\begin{align}
\mathbbf{T}(\mathbbf{T}_1(\vec{C}))=\vec{C}\\
\text{and  }\mathbbf{T}_1(\mathbbf{T}(\vec{A}))=\vec{A}\\
\text{Let }\mathbbf{T}(\vec{A})=\vec{C}\\
\implies \mathbbf{T}_1(\vec{C})=\vec{A}\\
\implies \vec{B}(\mathbbf{T}_1(\vec{C}))=\vec{B}\vec{A}=\vec{C}
\end{align}
So,the inverse transformation matrix is $\vec{B}^{-1}$\\
Consider the following
\begin{align}
  \mathbbf{T}_1(\vec{C})=\vec{B}^{-1}(\vec{B}\vec{A})=\vec{A}\\
  \implies \vec{B}^{-1}\vec{B}=\vec{I}_{m \times m}\label{3}\\
  \mathbbf{T}(\vec{A})=\vec{B}(\vec{B}^{-1}\vec{C})=\vec{C}\\
  \implies \vec{B}\vec{B}^{-1}=\vec{I}_{p \times p}\label{4}
\end{align}
where $\vec{I}$ is the identity matrix.\\
But
\begin{align}
    \vec{B}\vec{B}^{-1}=\vec{B}^{-1}\vec{B}=\vec{I}\label{2}
\end{align}
 So,from \eqref{3},\eqref{4},\eqref{2} 
\begin{align}
    p=m
\end{align}
So,$\vec{B}$ is an invertible $m \times m$ matrix
\subsection{Case 2}
 Consider $p=m$ and $\vec{B}$ is an invertible $m \times m$ matrix.\\
Verifying if $\mathbbf{T}$ is onto,\\
Let the set of matrices \cbrak{\vec{A}_1,\vec{A}_2,\hdots,\vec{A}_{mn}} be the basis for $\vec{V}$\\
Any matrix $\vec{A} \in \vec{V}$ can be written as
\begin{align}
   \vec{A}= \sum_{i=1}^{mn} \alpha_i\vec{A}_i\label{5}
\end{align}
where $\alpha_i \in \mathbbf{F}$\\
The set $\vec{M}= $ \cbrak{\vec{B}\vec{A}_1,\vec{B}\vec{A}_2,\hdots,\vec{B}\vec{A}_{mn}} lie in $\vec{W}$
\begin{align}
    c_1(\vec{B}\vec{A}_1)+c_2(\vec{B}\vec{A}_2)+\hdots+c_{mn}(\vec{B}\vec{A}_{mn})=\vec{0}\\
    \implies \vec{B}(c_1\vec{A}_1+c_2\vec{A}_2+\hdots+c_{mn}\vec{A}_{mn})=\vec{0}
\end{align}
Since $\vec{B}$ is non-singular,
\begin{align}
    (c_1\vec{A}_1+c_2\vec{A}_2+\hdots+c_{mn}\vec{A}_{mn})=\vec{0}\\
    \implies c_1,c_2,\hdots,c_{mn}=0
\end{align}
because \cbrak{\vec{A}_1,\vec{A}_2,\hdots,\vec{A}_{mn}} are linearly independent\\
So,$\vec{M}$ forms basis for $\vec{W}$\\
Any vector $\vec{C} \in \vec{W}$ can be written as
\begin{align}
    \vec{C}=\sum_{i=1}^{mn} \beta_i\vec{B}\vec{A}_i \text{  where }\beta_i \in \mathbbf{F}\\
    =\vec{B}(\sum_{i=1}^{mn}\beta_i\vec{A}_i)\\
    =\vec{B}\vec{A} \text{  (from \eqref{5} )}
\end{align}
So,range($\vec{B}$)=$\vec{W}$\\
Consider the matrix $\vec{A},\vec{A'} \in \vec{V}$ such that
\begin{align}
    \vec{B}\vec{A}= \vec{B}\vec{A'}\\
    \vec{B}^{-1}( \vec{B}\vec{A})= \vec{B}^{-1}(\vec{B}\vec{A'})\\
    (\vec{B}^{-1} \vec{B})\vec{A}= (\vec{B}^{-1}\vec{B})\vec{A'}\\
    \implies \vec{A}=\vec{A'}
\end{align}
So, $\mathbbf{T}$ is invertible.
\subsection{Conclusion}
From case 1,case 2 $\mathbbf{T}$ is invertible if and only if $p=m$ and $\vec{B}$ is an invertible $m \times m$ matrix.
\subsection{Example}
Let $p=m=3 ,n=4$
Let $\mathbbf{T}:\vec{V} \rightarrow \vec{W}$ adds row 2 to row 3 for a matrix $\vec{A} \in \vec{V}$\\
The elementary matrix that performs this is
\begin{align}
    \vec{B}= \myvec{1&0&0\\0&1&0\\0&1&1}
\end{align}
\begin{align}
    \text{Let }\vec{A}=\myvec{1&2&2&5\\1&3&6&7\\4&9&2&6}\\
    \mathbbf{T}(\vec{A})=\vec{B}\vec{A}\\=
    \myvec{1&0&0\\0&1&0\\0&1&1}\myvec{1&2&2&5\\1&3&6&7\\4&9&2&6}\\
    =\myvec{1&2&2&5\\1&3&6&7\\5&12&8&13}\\
    =\vec{C} \in \vec{W}
\end{align}
Let transformation $\mathbbf{T}_1:\vec{W} \rightarrow \vec{V}$ subtracts row2 from row 3 for a matrix $\vec{C} \in \vec{W}$ and is performed by elementary matrix
\begin{align}
\vec{U}=\myvec{1&0&0\\0&1&0\\0&-1&1}\\
     \text{Let }\vec{C}=\myvec{1&2&2&5\\1&3&6&7\\5&12&8&13}\\
     \mathbbf{T}_1(\vec{C})=\myvec{1&0&0\\0&1&0\\0&-1&1}\myvec{1&2&2&5\\1&3&6&7\\5&12&8&13}\\
     =\myvec{1&2&2&5\\1&3&6&7\\4&9&2&6}\\=\vec{A}\\
     \implies \mathbbf{T}_1(\vec{C})=\vec{A}\\
     \mathbbf{T}_1(\mathbbf{T}(\vec{A}))=\vec{A}\\
     \text{and  } \mathbbf{T}(\vec{A})=\vec{C}\\
     \implies \mathbbf{T}(\mathbbf{T}_1(\vec{C}))=\vec{C}
     \end{align}
     So,$\mathbbf{T}_1$ is the inverse transformation of $\mathbbf{T}$ and
     \begin{align}
     \mathbbf{T}_1=\mathbbf{T}^{-1}\\
     \text{Also,  } \vec{U}\vec{B}=\myvec{1&0&0\\0&1&0\\0&-1&1}\myvec{1&0&0\\0&1&0\\0&1&1}\\
     =\myvec{1&0&0\\0&1&0\\0&0&1}\\
     \vec{B}\vec{U}=\myvec{1&0&0\\0&1&0\\0&1&1}\myvec{1&0&0\\0&1&0\\0&-1&1}\\=\myvec{1&0&0\\0&1&0\\0&0&1}\\
     \implies \vec{B}^{-1}=\vec{U}
\end{align}
So, $\mathbbf{T}$ is invertible and ,$\vec{B}$ is an invertible $3 \times 3$ matrix.
\end{document}

